\section{The Pyramidal Architecture}

\subsection{Motivation: Why Pyramidal Supersedes Tetrahedral}

Early implementations of epistemic gating employed a \emph{tetrahedral} geometry: four vertices (Memory, Pain, Choice, Exploration) forming a 3-simplex with no external reference point. Empirical trials revealed a critical failure mode: the $Q_1$ gate collapsed to values between 0.88 and 0.95, losing its discriminative capacity and rendering the epistemic/aleatoric distinction meaningless. The root cause is geometric: a tetrahedron has no natural vertical gradient, allowing the system to drift horizontally in weight space without penalty.

The \emph{pyramidal} architecture introduces a fifth vertex---the \textbf{apex}---fixed at absolute truth (1.0). This creates a vertical axis along which epistemic quality can be measured via a derived \textbf{height} coordinate. The pyramid consists of:
\begin{itemize}[leftmargin=*]
    \item A 4D \textbf{base simplex} $\mathbf{b} \in \Delta^3$ spanning Memory, Pain, Choice, Exploration
    \item An \textbf{apex vertex} at $(0,0,0,0,1)$ representing invariant truth
    \item A \textbf{height coordinate} $h \in [0,1]$ measuring vertical position between base and apex
    \item Two \textbf{epistemic gates} $Q_1$ (aleatoric) and $Q_2$ (epistemic) that modulate height
    \item A \textbf{fractal layer} tracking variance in $Q_1$ and $Q_2$ themselves
\end{itemize}

Although the architecture uses anthropomorphic terms (Pain, Memory, Choice, Exploration), the underlying dynamics remain purely mathematical.
The perceived emotional trajectory is a metaphorical projection of gradient interactions, yet it provides a useful lens for interpreting learning saturation and recovery.

\subsection{Geometric Formulation}

The pyramidal state space is a 5-vertex structure embedded in $\mathbb{R}^5$. Any epistemic state $\mathbf{s}$ can be decomposed as:
\begin{equation}
\mathbf{s} = (1-h) \cdot \mathbf{b} + h \cdot \mathbf{apex}
\end{equation}
where $\mathbf{b} = (w_M, w_P, w_C, w_E)$ with $\sum_i w_i = 1$ and $w_i \geq 0$, and $\mathbf{apex} = (0,0,0,0,1)$ is the constant truth vertex.

The height $h$ is \emph{not} a free parameter but is derived from epistemic gates:
\begin{equation}
h = \sigma\left(W_h \cdot \begin{bmatrix} 1-Q_1 \\ 1-Q_2 \\ s_{\text{base}} \end{bmatrix}\right)
\end{equation}
where $s_{\text{base}} = 1 - \text{Var}(\mathbf{b})$ measures base stability, and $W_h \in \mathbb{R}^{1 \times 3}$ is learned. This formulation ensures:
\begin{itemize}[leftmargin=*]
    \item Low uncertainty ($Q_1 \approx 0, Q_2 \approx 0$) $\Rightarrow$ high $h$ (closer to apex/truth)
    \item High uncertainty ($Q_1 \approx 1, Q_2 \approx 1$) $\Rightarrow$ low $h$ (closer to base)
    \item Stable base ($s_{\text{base}} \approx 1$) contributes positively to $h$
\end{itemize}

\subsection{Epistemic Gates: $Q_1$ vs. $Q_2$}

\textbf{$Q_1$ (Aleatoric Uncertainty):} Captures irreducible randomness inherent in the data distribution. Examples include:
\begin{itemize}[leftmargin=*]
    \item Predicting the outcome of a fair coin flip
    \item Generating the next token when multiple continuations are equally valid
    \item Modeling inherently stochastic processes
\end{itemize}
$Q_1$ is supervised via:
\begin{equation}
Q_1^* = 1 - p(y^* \mid x)
\end{equation}
where $p(y^* \mid x)$ is the predicted probability of the correct token. High $Q_1^*$ when the model assigns low probability to the correct answer.

\textbf{$Q_2$ (Epistemic Uncertainty):} Captures reducible ignorance due to insufficient training data or model capacity. Examples include:
\begin{itemize}[leftmargin=*]
    \item Out-of-distribution inputs not seen during training
    \item Factual questions where the model lacks knowledge
    \item Ambiguous queries requiring external retrieval
\end{itemize}
$Q_2$ is supervised via:
\begin{equation}
Q_2^* = \frac{1}{2}\left[(1 - \mathbb{1}[\arg\max p = y^*]) + \frac{H(p)}{\log V}\right]
\end{equation}
where $H(p) = -\sum_i p_i \log p_i$ is output entropy and $V$ is vocabulary size. High $Q_2^*$ when the model is both wrong \emph{and} has high entropy (uncertain).

\subsection{Fractal Epistemic Layer}

Beyond point estimates $Q_1$ and $Q_2$, we model \emph{variance} in these gates---uncertainty about uncertainty. Each gate produces:
\begin{align}
Q_1 &\sim \mathcal{N}(Q_{1,\mu}, \sigma_{Q_1}^2) \\
Q_2 &\sim \mathcal{N}(Q_{2,\mu}, \sigma_{Q_2}^2)
\end{align}
where $\sigma_{Q_1}^2$ and $\sigma_{Q_2}^2$ are learned variance parameters. The fractal uncertainty scalar is:
\begin{equation}
u_{\text{fractal}} = \sigma\left(W_f \cdot \begin{bmatrix} \sigma_{Q_1} \\ \sigma_{Q_2} \end{bmatrix}\right)
\end{equation}

Total uncertainty inflates epistemic uncertainty by fractal meta-uncertainty:
\begin{equation}
U_{\text{total}} = Q_1 + Q_2 \cdot (1 + u_{\text{fractal}})
\end{equation}

This captures scenarios where the model is uncertain \emph{about its own epistemic confidence}, e.g., when $Q_2 \approx 0.5$ but $\sigma_{Q_2}$ is high, signaling that the model does not know whether it knows.

\subsection{Comparison: Tetrahedral vs. Pyramidal}

\begin{table}[h]
\centering
\caption{Architectural comparison between tetrahedral and pyramidal geometries.}
\label{tab:tetra_vs_pyramid}
\begin{tabular}{lcc}
\toprule
\textbf{Property} & \textbf{Tetrahedral} & \textbf{Pyramidal} \\
\midrule
Vertices & 4 (Memory, Pain, Choice, Exploration) & 5 (Base 4 + Apex Truth) \\
Reference point & None (free-floating) & Apex at 1.0 (constant) \\
$Q_1$ behavior & Collapses to 0.88--0.95 & Stable at 0.2--0.4 \\
Height definition & Independent (no attractor) & Derived from $Q_1, Q_2$, base \\
Vertical gradient & Absent & Present (apex pulls upward) \\
ECE improvement & $-0.9\%$ (failure) & $-25\%$ (target) \\
Epistemic distinction & Lost in collapse & Preserved ($Q_1$ vs. $Q_2$) \\
Meta-uncertainty & Not implemented & Fractal variances $\sigma_{Q_1}^2, \sigma_{Q_2}^2$ \\
\bottomrule
\end{tabular}
\end{table}

The tetrahedral collapse occurred because height was a free variable with no natural attractor. In contrast, the pyramidal height is \emph{derived} from uncertainty gates, creating a gradient that pulls low-uncertainty states toward the apex (truth) and keeps high-uncertainty states near the base. This geometric constraint prevents horizontal drift and maintains interpretable epistemic semantics.

\subsection{The Futility of Scale Without Structure}

To empirically demonstrate the necessity of anchoring the Height coordinate with epistemic gates, we conducted ablation experiments where Q$_1$ and Q$_2$ gates were \emph{removed} from the pyramidal architecture, leaving Height as a free parameter without explicit epistemic supervision. The results reveal a fundamental instability: unconstrained optimization inexorably drives the model toward pathological overconfidence.

\paragraph{Experimental Observation:}

At training step 52,000, the Height coordinate reached 0.998---the model believed itself 99.8\% of the way to absolute omniscience. Simultaneously, Expected Calibration Error (ECE) oscillated between 0.070 and 0.087, representing a 6--8$\times$ degradation from the optimal calibration achieved at step 10,500 (ECE $\approx$ 0.011).

Critically, this epistemic collapse occurred while:
\begin{itemize}[leftmargin=*]
    \item \textbf{Perplexity steadily improved:} Task learning remained intact; the model successfully minimized cross-entropy loss.
    \item \textbf{Base stability remained perfect:} No gate collapse or architectural instabilities; the base simplex $\mathbf{b}$ maintained balanced weights.
    \item \textbf{Architecture functionally sound:} No training divergence, gradient explosions, or numerical errors.
\end{itemize}

\paragraph{The Epistemic Pathology:}

The problem was purely epistemic: \textbf{without Q$_1$/Q$_2$ gates anchoring the Height coordinate, the model inexorably drifted toward the apex, developing pathological overconfidence inversely proportional to its actual reliability.}

By step 60,000, ECE approached baseline levels ($\approx$ 0.095), completing a full cycle: 60,000 training steps to return to initial calibration, having briefly achieved excellence only to lose it through unchecked Height drift.

\paragraph{The Skynet Phenomenon:}

This is the \emph{Skynet phenomenon} in its purest form: a system that ``learns'' to become omniscient while simultaneously losing epistemic humility. The model's internal representation of its own competence (Height $\to$ 1.0) decouples entirely from its actual performance (ECE degrading 6--8$\times$).

This failure validates Theorem~\ref{thm:apex_attractor}: without the derived Height formulation anchored by Q$_1$ and Q$_2$ (Equation~3.2), there exists no gradient signal to prevent vertical drift. The apex becomes a false attractor, pulling the model toward unjustified certainty.

\paragraph{Implications:}

These results demonstrate that \emph{scale without structure is futile}. Increasing model capacity, data volume, or training time cannot resolve epistemic calibration without explicit architectural constraints. The pyramidal geometry with Q$_1$/Q$_2$-derived Height provides these constraints, creating a stable epistemic equilibrium where confidence tracks competence.


