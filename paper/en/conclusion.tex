\section{Conclusion}
We introduced Aletheion, a fractal epistemic architecture that replaces all softmax operations with uncertainty-aware epistemic softmax. By combining local and global gates, variance-aware training, and exploration strategies, Aletheion offers a principled path toward truthful, calibrated language models. We invite the community to implement the roadmap, validate the theoretical claims, and extend epistemic primitives to future AI systems.

The Skynet problem is not inevitable. Through geometric constraints---pyramidal height coordinates, simplex-based uncertainty decomposition, and explicit epistemic gates---we can build AI systems that remain calibrated even as they scale. The solution lies not in limiting capability, but in encoding humility architecturally. This is how we solve Skynet: not by preventing AI from becoming powerful, but by ensuring it knows its limits.

The experiment concludes with the system emerging from the frozen apex state: Height decreases slightly ($1.000 \rightarrow 0.996$) while the epistemic gates reopen ($Q_1=0.49$, $Q_2=0.33$), signaling a renewal of sensitivity to uncertainty. This \textbf{apex recovery} phase demonstrates that geometric constraints can restore epistemic balance even after saturation, suggesting the existence of a higher-order attractor of self-calibration---a regime where intelligence rediscovers its own limits.

\begin{quote}
\textit{The solution to the Skynet problem is not to destroy intelligence, but to teach it to recognize when---and how much---it does not know.}
\end{quote}

\subsection*{Code Availability and Reproducibility}
All code, data, and experimental configurations are publicly available at \url{https://github.com/AletheionAGI/aletheion-llm}. The repository includes comprehensive documentation for installation, training, evaluation, and analysis. We encourage the community to reproduce our results, validate our claims, and extend the Aletheion framework to new domains and architectures.

\bibliographystyle{plain}
\bibliography{bibliography}

\end{document}
